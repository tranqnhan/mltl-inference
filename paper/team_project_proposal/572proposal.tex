\documentclass[12pt]{article}
% \documentclass{elsarticle} %A different option for format styling...

%NOTE: Use of this template is *totally optional* -- it is just provided to make your life easier. If it does not do that, feel free to use your own template. The formatting does not count for points, only the answers in a readable (LaTeX-generated) format count. 

\textwidth 6.5in
\oddsidemargin 0.0in %this is a 1-inch margin
\evensidemargin 1.0in %matching 1-inch margin

\usepackage{amssymb}
\usepackage{alltt}
\usepackage{multicol}
\usepackage{hyperref}
\usepackage{mathrsfs} %for \mathscr{} 
\usepackage{amsthm}
\usepackage{amsmath}
\usepackage{listings}
\usepackage{gensymb} %for \degree
\usepackage{longtable} %for longtabu
\usepackage{hhline} %for double \hline in longtabu
\usepackage{blindtext}

\newtheorem{defin}{Definition}
\newtheorem{intuit}{Intuition}

%Here are the commands included in elsarticle style:
\newtheorem{thm}{Theorem}
\newtheorem{lem}[thm]{Lemma}

\interfootnotelinepenalty=10000

\renewcommand{\phi}{\varphi}
\newcommand{\always}{\Box}
\newcommand{\eventually}{\Diamond}
\newcommand{\calL}{{\cal L}}

\newcommand{\pspic}[2]{\scalebox{#1}{\includegraphics{#2}}}



%Number the Exercises with one counter through multiple sections
\newcounter{ExerciseCounter}
\setcounter{ExerciseCounter}{1} %start counting at 1


%%%%%%%%%%%%%%%%%%%%%%%%%%%%%%%%%%%%%%%%%%%%%%%%%%%%%%%%
% Figure Magic
%%%%%%%%%%%%%%%%%%%%%%%%%%%%%%%%%%%%%%%%%%%%%%%%%%%%%%%%
\usepackage{epsfig}
\usepackage{float}
\usepackage{subfigure}
\usepackage{wrapfig}
\renewcommand{\topfraction}{.95} %figures can take up at most 95% of the page before being alone
\renewcommand{\bottomfraction}{.99} %figures can take up at most 99% of the page before being alone
\renewcommand{\textfraction}{.1} %at most this this % of page will be text before making figure-only page
\addtolength{\abovecaptionskip}{-3mm}


\begin{document}

\title{\bf
\large COM S 572 : Principles of Artificial Intelligence \\
\large MLTL Inference -- Proposal}

\author{
  Luke Marzen\\
  \texttt{ljmarzen@iastate.edu}
  \and
  Jayaraman, Swaminathan\\
  \texttt{swamjay@iastate.edu}
  \and
  Nhan Tran\\
  \texttt{nhtran@iastate.edu}
  \and
  Zili Wang\\
  \texttt{ziliw1@iastate.edu}
}

\date{March 8, 2024}

\maketitle

% INSTRUCTIONS FROM CANVAS
% Each group submits via Canvas a proposal that contains the title of your project, a brief (a few paragraphs) outline of the project, and a list of group members as well as the role of each member
% <https://canvas.iastate.edu/courses/108080/assignments/2206881>

Mission-time Linear Temporal Logic (MLTL) is a discrete time, finite interval bounded temporal logic that has found numerous recent applications. 
For example, MLTL was the specification logic for NASA's Robonaut2 verification project \cite{KZJZR20}, as well as for the design-time and runtime verificaiton of the NASA Lunar Gateway Vehicle System Manager \cite{DBR21}.
Other applications of MLTL include autonomous satellite \cite{JAXA}, UAV Traffic Management \cite{HCHJR21}, and more. 

Given a (finite) set of atomic propositions $\mathcal{AP}$, the syntax of MLTL formulas $\phi$ and $\psi$ are defined recursively: 
$$ \phi, \psi := true \ | \ false \ | \ p \ | \ \neg \phi \ | \ \phi \land \psi \ | \ \phi \lor \psi \ | \ F_{[a,b]} \phi \ | \ G_{[a,b]} \phi \ | \ \phi U_{[a,b]} \psi \ | \ \phi R_{[a,b]} \psi, $$
where $p \in \mathcal{AP}$, and $a, b \in \mathbb{Z}$ such that $0 \leq a \leq b$. 
The symbols $F, G, U, R$ denote the temporal operators Future, Globally, Until, and Release, respectively.
A trace $\pi$ represents a sequence of truth assignments to the atomic propositions in $\mathcal{AP}$ over time, and $\pi$ satisfies a MLTL formula $\phi$ is denoted as $\pi \models \phi$.
If $\pi$ does not satisfy $\phi$, then $\pi \not\models \phi$.
The explicit semantics of each MLTL operator can be found in \cite{WEST-iFM23}.

The object of this proposed project is to explore various approaches to MLTL inference, formally described as follows: 
Given a set traces $T = T^+ \cup T^-$ over $n$ variables, learn a MLTL formula $\phi$ such that for all $\pi \in T^+$, $\pi \models \phi$, and for all $\pi \in T^-$, $\pi \not\models \phi$. 
Intuitively, $T^+$ represents the set of positive examples, or desirable behaviors, and $T^-$ represents the set of negative examples, or undesirable behaviors.
These traces can come from simply observing the behavior of some system of interest, or they can be generated by some other means, such as a model checker or a simulator.
The goal is to learn a MLTL formula that specifies the desired behavior of the system, and to do so in a way that generalizes to new, unseen traces.

There is a large corpus of work on learning regular languages, stemming from Angluin's $L^*$ algorithm \cite{ANGLUIN_Lstar} that learns a minimal deterministic finite automata (DFA) from positive and negative examples. Work in learning temporal logic formulas have also recently been explored, with approaches ranging from symbolic learning algorithms \cite{roy_ltlf_learning, camacho_ltlf_learning} to deep learning algorithms \cite{stl_learning, Luo_Liang_Du_Wan_Peng_Zhang_2022}.
There is no published work on learning MLTL formulas, and the goal of this project is to explore various approaches to this problem, and to compare and contrast their performance.

Previously, Zili Wang has developed and evaluated Genetic Programming (GP) based approach to learning MLTL formulas, and lays the groundwork for the proposed project.
The repository containing the code and datasets for this project can be found at 

\noindent \url{https://github.com/zwang271/mltl-inference}.
Work that will be reused includes the following components:
\begin{enumerate}
  \item A parser (Python) for MLTL formulas, that can compute various properties of the formula, such as the number of atomic propositions, the syntax tree, worst propagation delay (wpd) (defined in \cite{KZJZR20}), and various other useful functions.  
  \item An MLTL interpreter (C++): on input trace $\pi$ and a MLTL formula $\phi$, determines if $\pi \models \phi$.
  \item A dataset generator (Python) that given an MLTL formula, uses either random trace sampling or regular expression sampling (see \cite{WEST-iFM23}) to generate a set of positive and negative examples.
  \item 9 datasets, each with 500 positive and 500 negative examples, split into 80\% training and 20\% testing sets.
\end{enumerate}

As a group, we aim evaluate various approaches using a combination of metrics such as run time, formula accuracy as a percent of correctly classified traces, and formula length/complexity.  
Each team member will be responsible for implementing a different technique, though if some techniques are found to be non-viable or require more effort than reasonable to complete in a semester we will remain flexible and collaborate to ensure the project is completed.
A description of each approach and the team member responsible for it is as follows:
\begin{itemize}
  \item Zili - Graph Neural Network with weight extraction
  
  Prior work applies Graph Neural Networks (GNN) to learn Linear Temporal Logic over finite trace (LTLf) formulas by extracting formula from the weights of trained GNNs \cite{Luo_Liang_Du_Wan_Peng_Zhang_2022}.
  The approach will be adapted to MLTL, and we seek generalize the methodology in interesting ways. 
  For example, we may explore the use of different GNN architectures, or applying deep reinforcement learning techniques to a game playing reformulation of the problem. 

  \item Nhan - Transformer Neural Network
	
	Transformer Neural Network is a subset of Recurrent Neural Network. Transformers are popular in Natural Language Processing for their ability to identify context in text. In terms of this project, transformer will be used for its positional embedding through Attention and Self-attention mechanisms to identify the relationship between each token in the MLTL traces and its formula. The input to the transformer will be the MLTL traces. The output will be the MLTL formula.
    
	\item Swaminathan - Template driven search
	
  A template-driven approach for Mission Time Linear Temporal Logic (MLTL) inference uses predefined formula templates to guide the search for an MLTL formula consistent with given positive and negative traces. In this project, candidate MLTL formulas will be systematically generated by filling template placeholders through an automated search mechanism, prioritizing the efficiency of the search process. Each candidate is then evaluated against a set of positive and negative traces. Formula selection will be performed based on additional considerations such as simplicity and generalizability to ensure practical utility.

    \item Automata learning
    \item A* search
\end{itemize} 

\newpage
\bibliographystyle{plain} % We choose the "plain" reference style
\bibliography{Citations} % Entries are in the refs.bib file
\end{document}
